\documentclass{tufte-handout}

\title{Graph Theory 1MA170: Course information}

\author[Vilhelm Agdur]{Vilhelm Agdur\thanks{\href{mailto:vilhelm.agdur@math.uu.se}{\nolinkurl{vilhelm.agdur@math.uu.se}}}}

\date{2 September 2023}

%\geometry{showframe} % display margins for debugging page layout

\usepackage{graphicx} % allow embedded images
  \setkeys{Gin}{width=\linewidth,totalheight=\textheight,keepaspectratio}
  \graphicspath{{graphics/}} % set of paths to search for images
\usepackage{amsmath}  % extended mathematics
\usepackage{booktabs} % book-quality tables
\usepackage{units}    % non-stacked fractions and better unit spacing
\usepackage{multicol} % multiple column layout facilities
\usepackage{lipsum}   % filler text
\usepackage{fancyvrb} % extended verbatim environments
  \fvset{fontsize=\normalsize}% default font size for fancy-verbatim environments

\usepackage{color,soul} % Highlights for text

% Standardize command font styles and environments
\newcommand{\doccmd}[1]{\texttt{\textbackslash#1}}% command name -- adds backslash automatically
\newcommand{\docopt}[1]{\ensuremath{\langle}\textrm{\textit{#1}}\ensuremath{\rangle}}% optional command argument
\newcommand{\docarg}[1]{\textrm{\textit{#1}}}% (required) command argument
\newcommand{\docenv}[1]{\textsf{#1}}% environment name
\newcommand{\docpkg}[1]{\texttt{#1}}% package name
\newcommand{\doccls}[1]{\texttt{#1}}% document class name
\newcommand{\docclsopt}[1]{\texttt{#1}}% document class option name
\newenvironment{docspec}{\begin{quote}\noindent}{\end{quote}}% command specification environment

\include{mathcommands.extratex}

\begin{document}

\maketitle% this prints the handout title, author, and date

\begin{abstract}
\noindent
This file is intended to contain all the practical information you may need about the course -- when is the exam, what is the assignment, when are the lectures, and so on.\sidenote{If you notice some information is missing, please do tell me and I will add it.}
\end{abstract}

The course literature for this course is Reinhard Diestel's \emph{Graph Theory}\sidenote{Which is available in a free pdf format from the university library. \hl{Add link to the book.} \hl{Add citation to the book.}} and ``notes from the lecturer''. The course will of course be closer to the notes than to the book, but the book should still be a useful reference or alternative perspective -- though I do not guarantee that everything the lecture notes cover will be in the book. Since the course has a new lecturer this year\sidenote{The previous one defended his PhD and is no longer at Uppsala.}, ``notes'' could really refer both to the old notes and the new ones. \hl{Update this bit closer to the beginning of the course to reflect the state of the notes.}

\section{Lecture plan}

There will be a total of twenty scheduled sessions, of which \hl{how many} will be lectures, and the rest will be exercise sessions. The exercise sessions are important -- we will use them to introduce new concepts, and I \emph{will} assume in the lectures that you have been at the exercise sessions as well.\sidenote[][]{Of course, we all sometimes have to miss a lecture or exercise session. Attendance is, as always, voluntary. However, just as you would read the lecture notes to catch up on a missed lecture, you should attempt the exercises to catch up on a missed exercise session!}

The exact planning of the content of the lectures is still subject to change. \hl{Update this to reflect the state of the notes nearer to the start of the course.} \hl{Add in reminder to register for the exam.}

\begin{table}[h]
\begin{tabularx}{\textwidth}{llX}
L\# & Date \& Time      & Content \\ 
\midrule
1  & ??? & L1: Introduction -- what is graph theory?\\
2  & ??? & L2: Simple graphs and subgraphs\\
3  & ??? & L3: Trees\\
4  & ??? & L4: Counting spanning trees\\
5  & ??? & L5: Weights and distances\\
6  & ??? & L6: Hamilton cycles\\
7  & ??? & L7: ???\\
8  & ??? & L8: The max-flow min-cut theorem\\
9  & ??? & L9: Matchings\\
10 & ??? & L10: Connectivity\\
11 & ??? & L11: Planarity\\
12 & ??? & L12: Vertex colourings\\
13 & ??? & L13: More on colourings\\
14 & ??? & L14: Edge-colourings and Ramsey theory\\
15 & ??? & L15: ???\\
16 & ??? & L16: Szemerédi's regularity lemma\\
17 & ??? & L17: The Rado graph\\
18 & ??? & L18: The Erd\H{o}s-Rényi random graph\\
19 & ??? & L19: More on random graphs\\
20 & ??? & L20: ???
\end{tabularx}
\end{table}

\section{The exam}

The ordinary exam for the course is on \hl{when} -- remember to register at least twelve days in advance, i.e. by the \hl{when}, in order to get to write it. Studying for an exam and then not getting to write it is pretty dispiriting.\sidenote{This may or may not have happened to me once or twice during my undergrad and master's...} The exam corresponds to 2hp out of the total of 5hp that the course consists of.

There will be reexams for the course \hl{when} and \hl{when}.

\section{The hand-in assignment}

The course has a \emph{mandatory} hand-in assignment, which corresponds to three out of the total of five hp of the course.

\bibliography{references}
\bibliographystyle{plainnat}



\end{document}
