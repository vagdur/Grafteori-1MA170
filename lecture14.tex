\documentclass[nobib]{tufte-handout}

\title{Lecture 14: The probabilistic method and the Erd\H{o}s-Rényi random graph $\cdot$ 1MA020}

\author[Vilhelm Agdur]{Vilhelm Agdur\thanks{\href{mailto:vilhelm.agdur@math.uu.se}{\nolinkurl{vilhelm.agdur@math.uu.se}}}}

\date{28 November 2023}


%\geometry{showframe} % display margins for debugging page layout

\usepackage{graphicx} % allow embedded images
  \setkeys{Gin}{width=\linewidth,totalheight=\textheight,keepaspectratio}
  \graphicspath{{graphics/}} % set of paths to search for images
\usepackage{amsmath}  % extended mathematics
\usepackage{booktabs} % book-quality tables
\usepackage{units}    % non-stacked fractions and better unit spacing
\usepackage{multicol} % multiple column layout facilities
\usepackage{lipsum}   % filler text
\usepackage{fancyvrb} % extended verbatim environments
  \fvset{fontsize=\normalsize}% default font size for fancy-verbatim environments

\usepackage{color,soul} % Highlights for text

% Standardize command font styles and environments
\newcommand{\doccmd}[1]{\texttt{\textbackslash#1}}% command name -- adds backslash automatically
\newcommand{\docopt}[1]{\ensuremath{\langle}\textrm{\textit{#1}}\ensuremath{\rangle}}% optional command argument
\newcommand{\docarg}[1]{\textrm{\textit{#1}}}% (required) command argument
\newcommand{\docenv}[1]{\textsf{#1}}% environment name
\newcommand{\docpkg}[1]{\texttt{#1}}% package name
\newcommand{\doccls}[1]{\texttt{#1}}% document class name
\newcommand{\docclsopt}[1]{\texttt{#1}}% document class option name
\newenvironment{docspec}{\begin{quote}\noindent}{\end{quote}}% command specification environment

\include{mathcommands.extratex}

\begin{document}

\maketitle% this prints the handout title, author, and date

\begin{abstract}
\noindent
We introduce a new tool to the course, the probabilistic method, and use it to prove some new results. We also introduce the Erd\H{o}s-Rényi random graph, which is interesting in its own right as well as being a tool in the probabilistic method.
\end{abstract}

\section{Girth and chromatic number}

Already when we first introduced the chromatic number, we mentioned the obvious fact that a large clique forces a graph to have high chromatic number. One might be led by this to believe that being very sparse, without any cliques, would mean you have low chromatic number. This unfortunately turns out to be not at all true, in a pretty strong sense.

\begin{definition}
  The \emph{girth} of a graph $G$ is the length of the shortest cycle in the graph.
\end{definition}

So the notion of girth generalizes the notion of being triangle-free, which is just having girth greater than three. Of course any triangle-free graph also contains no larger cliques, so this is indeed a strong notion of having no cliques.

\begin{theorem}[Erd\H{o}s, 1959]
  For all positive integers $k$ there exists a graph $G$ of girth and chromatic number at least $k$.
\end{theorem}

The proof of this uses the probabilistic method, and proceeds by considering a random graph from a suitably chosen distribution, and showing that the probability that it has girth and chromatic number at least $k$ is greater than $0$. We postpone the proof of this until a later lecture, when we will have developed our probabilistic toolbox a bit more.


\section{Exercises}


%\bibliography{references}
%\bibliographystyle{plainnat}

\end{document}
