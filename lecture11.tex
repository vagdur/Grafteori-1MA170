\documentclass[nobib]{tufte-handout}

\title{Lecture 11: Planarity $\cdot$ 1MA020}

\author[Vilhelm Agdur]{Vilhelm Agdur\thanks{\href{mailto:vilhelm.agdur@math.uu.se}{\nolinkurl{vilhelm.agdur@math.uu.se}}}}

\date{22 November 2023}


%\geometry{showframe} % display margins for debugging page layout

\usepackage{graphicx} % allow embedded images
  \setkeys{Gin}{width=\linewidth,totalheight=\textheight,keepaspectratio}
  \graphicspath{{graphics/}} % set of paths to search for images
\usepackage{amsmath}  % extended mathematics
\usepackage{booktabs} % book-quality tables
\usepackage{units}    % non-stacked fractions and better unit spacing
\usepackage{multicol} % multiple column layout facilities
\usepackage{lipsum}   % filler text
\usepackage{fancyvrb} % extended verbatim environments
  \fvset{fontsize=\normalsize}% default font size for fancy-verbatim environments

\usepackage{color,soul} % Highlights for text

% Standardize command font styles and environments
\newcommand{\doccmd}[1]{\texttt{\textbackslash#1}}% command name -- adds backslash automatically
\newcommand{\docopt}[1]{\ensuremath{\langle}\textrm{\textit{#1}}\ensuremath{\rangle}}% optional command argument
\newcommand{\docarg}[1]{\textrm{\textit{#1}}}% (required) command argument
\newcommand{\docenv}[1]{\textsf{#1}}% environment name
\newcommand{\docpkg}[1]{\texttt{#1}}% package name
\newcommand{\doccls}[1]{\texttt{#1}}% document class name
\newcommand{\docclsopt}[1]{\texttt{#1}}% document class option name
\newenvironment{docspec}{\begin{quote}\noindent}{\end{quote}}% command specification environment

\include{mathcommands.extratex}

\begin{document}

\maketitle% this prints the handout title, author, and date

\begin{abstract}
\noindent
We study the notion of a graph being \emph{planar}, define its \emph{planar dual}, and prove some results about when a graph is planar. We give most of a proof of the theorems of Kuratowski and Wagner about forbidden minors for planar graphs. Finally, we introduce the notion of outerplanar graphs.
\end{abstract}

\section{Three-connected graphs}

While the topic of this lecture is planarity, we will need a result about the structure of three-connected graphs. So we begin by proving this.

\begin{figure}
  \centering
  \includegraphics[width=0.75\textwidth]{graphics/L11_planarity/edge_contraction.png}
  \caption[][0cm]{A graph with an edge between $x$ and $y$ highlighted in red. On the right, the result of contracting this edge.}
  \label{fig:edge_contraction}
\end{figure}

\begin{definition}
  Let $G = (V,E)$ be a graph and $e = x\sim y$ be an edge of $G$. The \emph{contraction} of $e$ is the graph $G/e$, which we construct as follows:

  Introduce a new vertex $v_{xy}$, and let $V(G/e) = (V \setminus \{x,y\}) \cup \{v_{xy}\}$. Replace each edge $w \sim x$ or $w \sim y$ with an edge $w \sim v_{xy}$.
\end{definition}

\begin{lemma}
  If $G = (V,E)$ is three-connected and has more than four vertices, then there exists an edge $e \in E$ such that $G/e$ is again three-connected.

  \begin{proof}
    Suppose there is no such edge, so that for every edge $e = x \sim y$ there is a separating set $X$ of $G/e$ on two or fewer vertices. Now, since $G$ is three-connected, we must in fact have $X = \{v_{xy}, z\}$.

    Then the set $\{x, y, z\}$ must be a separating set for $G$. Each of these three vertices must have a neighbour in every connected component of $G[V \setminus \{x,y,z\}]$.\sidenote[][]{
      \begin{xca}
        Prove this.
      \end{xca}
    } Let $C$ be the smallest such component, and further assume we chose $x$, $y$, and $z$ in such a way that $\abs{V(C)}$ was minimal across all choices.

    Now, choose a neighbour $v$ of $z$ in $C$. By our assumption, no edge-contracton is three-connected, so in particular $G/(v \sim z)$ is not three-connected. By entirely the same argument as before, we can find a vertex $w$ such that $\{z,v,w\}$ separates $G$.

    Again, each of $z$, $v$, and $w$ has a neighbour in each connected component of $G[V \setminus \{z, v, w\}]$. Since there is an edge between $x$ and $y$, they must be in the same component, and so there is a component $D$ which contains neither. Since $v \in C$, each of its neighbours is also in $C$, and so in particular is its neighbour in $D$.

    So $D \cap C \neq \emptyset$. Since $D$ does not contain any of $x$, $y$, or $z$, their removal can't disconnect $D$, and so it follows from this that $D \subseteq C$. However, $D$ clearly does not contain $v$, being a connected component of $G[V \setminus \{z, v, w\}]$, while $C$ does contain $v$. So $D$ is in fact strictly smaller than $C$, which is a contradiction, since we assumed $C$ was minimal. So the lemma follows.
  \end{proof}
\end{lemma}

For our final theorem, we do not offer a proof, just a statement. It is in some sense analogous to our theorem about two-connected graphs being constructible by adding paths to a cycle graph.

\begin{theorem}[Tutte, 1961]
  A graph $G$ is three-connected if and only if there exists a sequence $G_0, G_1,\ldots, G_n$ of graphs, where $G_0 = K_4$, $G_n = G$, and for every $i \in [n]$ the graph $G_i$ has an edge $u \sim v$ with $d_{G_i}(u), d_{G_i}(v) \geq 3$ and $G_{i-1} = G_i/(u\sim v)$.
\end{theorem}

\section{Planarity}

One way to think about the notion of planarity is that we want to study graphs arising from maps, as we saw in the exercise session. Another might be that we just want to find nice ways to draw graphs. Let us give a not very precise definition of what we mean by drawing a graph:

\begin{definition}
  A \emph{drawing} or \emph{embedding} of a graph in $\R^2$ is a way of placing all the vertices of the graph at distinct points, and drawing an arc\sidenote[][]{In a more precise definition we would have to state what we assume about these arcs -- are they continuous, smooth, piecewise linear?} between any two adjacent vertices.

  We say that an embedding is \emph{planar}, and call the embedded graph a \emph{plane} graph, if this can be done so that none of the edges intersect.
\end{definition}

\begin{definition}
  Any planar embedding of a graph divides the plane into several connected components, called \emph{faces}. Exactly one of these faces is unbounded, and any other faces are bounded.\sidenote[][]{In order to prove this you need the seemingly obvious but surprisingly hard-to-prove Jordan curve theorem.}

  Generally, any cycle $C$ in a plane graph $G = (V,E)$ will separate the vertices of $G$ into three sets,
  $$V = O \amalg V(C) \amalg I,$$
  where $O$ are the vertices outside the cycle and $I$ the vertices inside the circle. There are no edges between the sets $O$ and $I$.
\end{definition}

Now we can give a definition of the notion of planar dual, which we introduced in the exercises:

\begin{definition}
  Let $G = (V,E)$ be a plane graph, that is, a planar graph with a given embedding. The \emph{planar dual} $G^* = (F, E^*)$ of $G$ is the multigraph whose vertices are the faces of the embedding of $G$, and where we draw an edge between two faces $f$ and $f'$ if they border each other.

  There is a natural one-to-one correspondence between $E$ and $E^*$ by sending each edge $e\in E$ to the edge between the two faces it separates.

  See Figure \ref{fig:planar_dual} for an illustration of this.
\end{definition}

\begin{figure}
  \centering
  \includegraphics[width=0.65\textwidth]{graphics/L11_planarity/planar_dual.png}
  \caption[][0cm]{A plane graph, with circular black vertices and grey edges, and its planar dual, with black crosses as vertices and red edges. Notice how the correspondence between $E$ and $E^*$ is on display here -- each red edge intersects precisely one grey edge, and vice versa.}
  \label{fig:planar_dual}
\end{figure}

Next, let us prove a theorem about what is often referred to as Euler's formula, though Euler in fact did not think about this for graphs but for polyhedra.

\begin{theorem}[Euler's formula]
  Let $G = (V,E)$ be a connected planar graph, and let $f$ be the number of faces for some planar embedding of $G$. Then\sidenote[][]{The thing on the left-hand side here is called the \emph{Euler characteristic}, which is the first step on a long journey of geometry and the genus of surfaces. If you want a book-length discussion of the history of this formula, and the history of the notion of mathematical proof in general, read Imre Lakatos \emph{Proofs and Refutations}.}
  $$\abs{V} - \abs{E} + f = 2,$$
  and so in particular any two planar embeddings have the same number of faces.

  \begin{definition}
    Let $G = (V,E)$ be a connected planar graph, fix a planar embedding of it, and construct its planar dual $G^* = (F, E^*)$. Next, fix a spanning tree $T = (V, E_T)$ of $G$, and consider its complement in the planar dual, $T* = (F, E^* \setminus E_T)$.

    Now, let us show that this $T^*$ is in fact a tree.\sidenote[][]{We already showed this in the exercise session, but let us repeat ourselves.} If $T^*$ were disconnected, one of its connected components would not contain the unbounded face, and this set of faces would be enclosed by a cycle of $T$. But $T$ has no cycles, since it is a tree.

    If $T^*$ contained a cycle, this cycle would necessarily enclose a vertex of $T$, which would be separated from the rest of $T$, and so $T$ would be disconnected, which is again impossible since $T$ is a tree.

    So, we have seen that $T$ is a tree on $\abs{V}$ vertices and thus has $\abs{V} - 1$ edges, and $T^*$ is a tree on $\abs{F}$ vertices and thus $\abs{F}-1$ edges. However, since $T^*$ has as its edges the complement of the edges of $T$, they must together have $\abs{E}$ edges. So we have shown that
    $$\left(\abs{V} - 1\right) + \left(\abs{F} - 1\right) = \abs{E},$$
    which rearranges to show the theorem.
  \end{definition}
\end{theorem}

\section{Exercises}


%\bibliography{references}
%\bibliographystyle{plainnat}

\end{document}
