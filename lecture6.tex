\documentclass[nobib]{tufte-handout}

\title{Lecture 6: Weights, distances, and minimum spanning trees $\cdot$ 1MA020}

\author[Vilhelm Agdur]{Vilhelm Agdur\thanks{\href{mailto:vilhelm.agdur@math.uu.se}{\nolinkurl{vilhelm.agdur@math.uu.se}}}}

\date{10 November 2023}


%\geometry{showframe} % display margins for debugging page layout

\usepackage{graphicx} % allow embedded images
  \setkeys{Gin}{width=\linewidth,totalheight=\textheight,keepaspectratio}
  \graphicspath{{graphics/}} % set of paths to search for images
\usepackage{amsmath}  % extended mathematics
\usepackage{booktabs} % book-quality tables
\usepackage{units}    % non-stacked fractions and better unit spacing
\usepackage{multicol} % multiple column layout facilities
\usepackage{lipsum}   % filler text
\usepackage{fancyvrb} % extended verbatim environments
  \fvset{fontsize=\normalsize}% default font size for fancy-verbatim environments

\usepackage{color,soul} % Highlights for text

% Standardize command font styles and environments
\newcommand{\doccmd}[1]{\texttt{\textbackslash#1}}% command name -- adds backslash automatically
\newcommand{\docopt}[1]{\ensuremath{\langle}\textrm{\textit{#1}}\ensuremath{\rangle}}% optional command argument
\newcommand{\docarg}[1]{\textrm{\textit{#1}}}% (required) command argument
\newcommand{\docenv}[1]{\textsf{#1}}% environment name
\newcommand{\docpkg}[1]{\texttt{#1}}% package name
\newcommand{\doccls}[1]{\texttt{#1}}% document class name
\newcommand{\docclsopt}[1]{\texttt{#1}}% document class option name
\newenvironment{docspec}{\begin{quote}\noindent}{\end{quote}}% command specification environment

\include{mathcommands.extratex}

\begin{document}

\maketitle% this prints the handout title, author, and date

\begin{abstract}
\noindent
Continuing from our previous discussion on the existence of spanning trees and counting of them, we come to the problem of actually finding them. We give two different algorithms for this. Finally we consider the problem of computing distances in a graph, and give an algorithm for this.
\end{abstract}

As we saw in the exercises in our previous session, the question of spanning trees becomes more interesting if we also add weights to the edges of our graph, since we can then consider the \emph{minimum} spanning trees.

We also saw that to do this, we need to restrict ourselves to finite graphs and only positive weights on the edges, or stupid things will happen. If we additionally assume that all weights are distinct, there is even a \emph{unique} minimal spanning tree.

\section{Exercises}

%\bibliography{references}
%\bibliographystyle{plainnat}

\end{document}
