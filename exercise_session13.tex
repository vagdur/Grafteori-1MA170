\documentclass[nobib]{tufte-handout}

\title{Exercise session 13: Edge-colourings, Ramsey theory, and the probabilistic method $\cdot$ 1MA020}

\author[Vilhelm Agdur]{Vilhelm Agdur\thanks{\href{mailto:vilhelm.agdur@math.uu.se}{\nolinkurl{vilhelm.agdur@math.uu.se}}}}

\date{27 November 2023}


%\geometry{showframe} % display margins for debugging page layout

\usepackage{graphicx} % allow embedded images
  \setkeys{Gin}{width=\linewidth,totalheight=\textheight,keepaspectratio}
  \graphicspath{{graphics/}} % set of paths to search for images
\usepackage{amsmath}  % extended mathematics
\usepackage{booktabs} % book-quality tables
\usepackage{units}    % non-stacked fractions and better unit spacing
\usepackage{multicol} % multiple column layout facilities
\usepackage{lipsum}   % filler text
\usepackage{fancyvrb} % extended verbatim environments
  \fvset{fontsize=\normalsize}% default font size for fancy-verbatim environments

\usepackage{color,soul} % Highlights for text

% Standardize command font styles and environments
\newcommand{\doccmd}[1]{\texttt{\textbackslash#1}}% command name -- adds backslash automatically
\newcommand{\docopt}[1]{\ensuremath{\langle}\textrm{\textit{#1}}\ensuremath{\rangle}}% optional command argument
\newcommand{\docarg}[1]{\textrm{\textit{#1}}}% (required) command argument
\newcommand{\docenv}[1]{\textsf{#1}}% environment name
\newcommand{\docpkg}[1]{\texttt{#1}}% package name
\newcommand{\doccls}[1]{\texttt{#1}}% document class name
\newcommand{\docclsopt}[1]{\texttt{#1}}% document class option name
\newenvironment{docspec}{\begin{quote}\noindent}{\end{quote}}% command specification environment

\include{mathcommands.extratex}

\begin{document}

\maketitle% this prints the handout title, author, and date

\begin{abstract}
\noindent
We consider the notion of an edge-colouring, and some results of Ramsey theory. We see how the probabilistic method can be used to prove various results in graph theory.
\end{abstract}

\section{Edge-colourings}

In our last lecture, we looked at the notion of a vertex colouring of a graph, and derived a few results about it. Of course, vertices are not the only thing we could be colouring -- we could also look at colouring the \emph{edges} of a graph.

\begin{definition}
    Let $G = (V,E)$ be a graph. A \emph{proper}\sidenote[][]{Unlike for vertex colourings, we will actually be interested in improper edge colourings more often than proper ones, so we choose the opposite convention of including the word proper and omitting the word improper for them.} \emph{$k$-edge-colouring} is a function $c: E \to [k]$ such that no two edges which are incident to each other (i.e. share an endpoint) are assigned the same colour. If we do not have this restriction on incident edges, we call it just an (improper) edge colouring.

    The \emph{edge-chromatic number} of $G$, denoted $\chi_1(G)$,\sidenote[][]{This is sometimes also called the \emph{chromatic index} of $G$. The $1$ in the notation indicates that edges are one-dimensional -- if we ever need to refer to both the chromatic number and the edge-chromatic number at the same time, we may thus denote the chromatic number by $\chi_0(G)$, since vertices are zero-dimensional. In some texts the edge-chromatic number is denoted by $\chi'(G)$, but $\chi$ and $\chi'$ look way too similar in \LaTeX\ and on a blackboard, so let us avoid that notation.
    
    \begin{xca}Given this little discussion, can you offer a definition of what $\chi_2(G)$ might refer to for a plane graph?\end{xca}} is the smallest integer $k$ such that $G$ has a proper $k$-edge-colouring.
\end{definition}

\begin{xca}
    We observed for vertex colourings that there are trivial bounds for it in terms of the clique number and the independence number, and that each colour class is an independent set.

    Now, observe that a proper edge-colouring of $G$ is just a vertex colouring on the line graph of $G$. Use this observation to derive bounds on $\chi_1(G)$. What are the colour classes of a proper edge-colouring, using a term we've already defined?
\end{xca}

In our study of vertex colourings, we proved that all planar graphs can be coloured with five colours, using a trick known as Kempe changes.\sidenote[][]{We defined exactly what these are in the previous exercise session.} You can do something similar for proper edge-colourings, considering edge-induced components of edges with two colours and swapping those.

\begin{xca}
    Use a trick like this to prove the following theorem by König:\sidenote[][]{This exercise is probably a little bit tough, but should be doable if you give it some time. It definitely isn't as quick as the other ones could be, though.}

    \begin{theorem}[König, 1916]
        For every bipartite graph $G$ with maximal degree $\Delta$, we have $\chi_1(G) = \Delta$.
    \end{theorem}
\end{xca}

\section{Ramsey theory}

Ramsey theory, named after remarkable British mathematician Frank P. Ramsey,\sidenote[][]{Seriously, read his Wikipedia page if you get bored, he was almost a modern day Newton in terms of being British and inventing tons of stuff across fields.} studies the question of how large a graph has to be in order to always contain a given structure. The simplest case is that of the eponymous \emph{Ramsey number}. Let us give two definitions of it:

\begin{definition}
    For any integer $k$, the $k$-th \emph{Ramsey number} $R(k)$ is the smallest integer $n$ such that every edge-colouring of $K_n$ contains a monochromatic $K_k$.\sidenote[][]{Clearly, we mean improper edge colourings here, since there is no proper $2$-edge-colouring of any $K_n$ other than $K_2$. By ``containing a monochromatic $K_k$'' we mean that for some colour $i$, the edge-induced subgraph $K_n\langle c^{-1}(i)\rangle$ contains a $k$-clique.}
\end{definition}

\begin{definition}
    For any integer $k$, the $k$-th \emph{Ramsey number} $R(k)$ is the smallest integer $n$ such that every graph on $n$ vertices contains either a $k$-clique or an independent set of size $k$.
\end{definition}

\begin{xca}
    Prove that the above two definitions are equivalent.
\end{xca}

Of course, it is a non-trivial fact that these Ramsey numbers in fact even exist -- apriori it could be the case that for some $k$, there exist arbitrarily big graphs that contain neither a $k$-clique nor an independent set of size $k$. We will prove that this is not the case later, using the probabilistic method. For now, let us just show that one Ramsey number is finite:

\begin{xca}
    Prove that $R(3) = 6$.
\end{xca}


%\bibliography{references}
%\bibliographystyle{plainnat}

\end{document}
