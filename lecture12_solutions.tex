\documentclass[a4paper]{article}

\usepackage[T1]{fontenc}
\usepackage[english]{babel}
\usepackage{amssymb, amsmath}
\usepackage{amsthm}
\usepackage[margin = 4cm]{geometry}

\setlength{\parindent}{0pt}

\newtheorem{exercise}{Exercise}
\newenvironment{solution}{\begin{proof}[Solution]\renewcommand{\qedsymbol}{}\ignorespaces}{\end{proof}}

\title{Solutions to Exercises in Lecture 12}
\author{Jón Hákon Garðarsson}
\date{\today}

\begin{document}
\maketitle

\begin{exercise}
Prove the second half of lemma 6.
\end{exercise}

\begin{solution}
If \(G\) has no two-connected blocks, then all of its blocks must be \(K_2\) or an isolated vertex.
These blocks are clearly all two-colourable.
Now use the algorithm in exercise 2 to complete the colouring.

If there are no edges, then we can clearly colour the graph with only one colour.
\end{solution}

\begin{exercise}
Give an algorithm that permutes the colours of the blocks.
\end{exercise}

\begin{solution}
The block graph forms a tree.
Pick a block as a root and keep its colouring.
Now until we hit the leaves, do the following.
\begin{enumerate}
	\item Permute the colours of each of the root's children to match the cut vertices of the root.
	\item Remove the root from the tree.
	\item Each of the children is a root of their own tree, repeat step 1 through 3 on all of them.
\end{enumerate}
This will never run into a conflict because the block graph has no cycles.
\end{solution}

\begin{exercise}
This bound is sometimes sharp.
Try to think of a planar graph that is not three-colourable.
Can you think of two examples, neither of which contains the other?
\end{exercise}

\begin{solution}
The graph \(K_4\) is an example.
Another one is the wheel with five spokes, \(W_5\).
Neither of these is contained in the other; there are no four vertices in \(W_5\) pairwise adjacent, and \(K_4\) cannot contain \(W_5\) because it has fewer vertices.
\end{solution}

\begin{exercise}
Prove that every tree is two-colourable.
\end{exercise}

\begin{solution}
Pick a vertex as the root and colour it.
Now until we hit the leaves, do the following.
\begin{enumerate}
	\item Colour all of the root's children the other colour.
	\item Remove the root from the tree.
	\item Each of the children is a root of their own tree, repeat step 1 through 3 on all of them.
\end{enumerate}
This will never run into a conflict because trees have no cycles.
\end{solution}

\begin{exercise}
What is the chromatic number of the hypercube graph in \(n\) dimensions?
\end{exercise}

\begin{solution}
The chromatic number is two for \(n \geq 1\).
We prove this by induction.
The base case is trivial.
Now remember that a \((n + 1)\)-cube is made out of two \(n\)-cubes by adding an edge between the corresponding vertices.
With our induction hypothesis, we can colour one of the copies with two colours.
Now colour the other one with the opposite colouring.
Then when we join the two \(n\)-cubes, we get a proper colouring.
This also shows that a hypercube is bipartite.
\end{solution}

\begin{exercise}
For any graph \(G\) with adjacency matrix \(A_G\), we have
\[
	\chi(G) \leq \lambda_\text{max}(A_G) + 1.
\]
\end{exercise}

\begin{solution}
Consider a minimal induced subgraph \(H\) with \(\chi(H) = \chi(G)\).
Let \(v\) be a vertex of \(H\) of minimal degree \(\delta(H)\), in \(H\).
Since \(H\) is minimal, we have that \(H \setminus \{v\}\) is \(\chi(G) - 1\) colourable.
Therefore, we must have
\[
	\delta(H) \geq \chi(G) - 1
\]
that is
\[
	\chi(G) \leq \delta(H) + 1
\]
because otherwise we could add \(v\) back in and get a \(\chi(G) - 1\) colouring of \(H\).
By lemma 12 we get
\[
	\chi(G)
	\leq \delta(H) + 1
	\leq \lambda_\text{max}(H) + 1
	\leq \lambda_\text{max}(G) + 1.
\]
\end{solution}

\end{document}
