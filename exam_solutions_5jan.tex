\documentclass[nobib]{tufte-handout}

\title{Solutions for the exam on January 5th 2024}

\author[Vilhelm Agdur]{Vilhelm Agdur\thanks{\href{mailto:vilhelm.agdur@math.uu.se}{\nolinkurl{vilhelm.agdur@math.uu.se}}}}

\date{5 January 2024}


%\geometry{showframe} % display margins for debugging page layout

\usepackage{graphicx} % allow embedded images
  \setkeys{Gin}{width=\linewidth,totalheight=\textheight,keepaspectratio}
  \graphicspath{{graphics/}} % set of paths to search for images
\usepackage{amsmath}  % extended mathematics
\usepackage{booktabs} % book-quality tables
\usepackage{units}    % non-stacked fractions and better unit spacing
\usepackage{multicol} % multiple column layout facilities
\usepackage{lipsum}   % filler text
\usepackage{fancyvrb} % extended verbatim environments
  \fvset{fontsize=\normalsize}% default font size for fancy-verbatim environments

\usepackage{color,soul} % Highlights for text

% Standardize command font styles and environments
\newcommand{\doccmd}[1]{\texttt{\textbackslash#1}}% command name -- adds backslash automatically
\newcommand{\docopt}[1]{\ensuremath{\langle}\textrm{\textit{#1}}\ensuremath{\rangle}}% optional command argument
\newcommand{\docarg}[1]{\textrm{\textit{#1}}}% (required) command argument
\newcommand{\docenv}[1]{\textsf{#1}}% environment name
\newcommand{\docpkg}[1]{\texttt{#1}}% package name
\newcommand{\doccls}[1]{\texttt{#1}}% document class name
\newcommand{\docclsopt}[1]{\texttt{#1}}% document class option name
\newenvironment{docspec}{\begin{quote}\noindent}{\end{quote}}% command specification environment

\include{mathcommands.extratex}

\begin{document}

\maketitle% this prints the handout title, author, and date

\begin{abstract}
\noindent
Questions one through eight were all of the form ``reproduce this result from the lecture notes'', so we give no solution for those. For questions nine and ten we give a sketch of a solution.
\end{abstract}

\section{Question 10}

\begin{enumerate}[label=\alph*)]
    \item This is true. The easiest example is just the complete graph on $k+1$ vertices.
    
    In fact, the stronger statement that for every $k$, there is a $k$-regular Hamiltonian graph on $n$ vertices for every sufficiently large $n$ is also true. One construction of an example of such a graph might go as follows:
    
    Pick a large $n$, and create a graph $G$ by starting with a cycle on $n$ vertices. Then, pick $k-2$ matchings on $G^c$, and add those edges to $G$ to make it Hamiltonian.\sidenote[][]{Why is this possible? If you picked $n$ large enough, Dirac's theorem tells you there is a Hamiltonian cycle in $G^c$, so picking every second edge of this gets you your first matching. Then, removing the edges of this matching from $G^c$ doesn't reduce the degree of any vertex by too much (since you picked $n$ very large, $n - 2 - k > n/2$), so there is still a Hamiltonian cycle by Dirac's theorem, and repeat until you have all $k$ matchings.} That this is graph is Hamiltonian is obvious, since the cycle the construction started with is a Hamiltonian cycle.

    \item This is false. Recall that we proved in the lecture notes that
    $$\abs{E} \leq 3\abs{V} - 6$$
    as a corollary to Euler's formula, by using a double counting argument counting edges incident to faces. Then, in our proof of the five-colour theorem, we used this to prove that the minimum degree of a planar graph can be at most five -- that is, every planar graph has a vertex whose degree is at most five. So there are no $k$-regular planar graphs for $k > 5$.

    \item This is true. Consider the graph with two vertices and $k$ edges between those two vertices -- clearly the spanning trees of this graph are gotten precisely by picking one of the edges.
    
    \item This is false -- but it fails only for $k = 2$! For $k = 1$, it is obviously true -- any graph which is itself a tree has only one spanning tree. For $k > 2$, consider the cycle graph on $k$ vertices: A spanning tree of this graph is gotten by removing one of the edges of the graph to render it acyclic.
    
    To see that it fails for $k = 2$, let $G$ be any graph with more than one spanning tree. That it has more than one spanning tree implies it is not a tree, so it contains a cycle, say $C$. Now pick a spanning forest $F$ of $G \setminus C$.\sidenote[][]{$G\setminus C$ may of course be disconnected, so we need to say spanning \emph{forest} here.} For any edge $e$ of $C$, we get a spanning tree of $G$ by taking $F \cup (C \setminus e)$ -- and any cycle of course has more than two edges, so this procedure gives us more than two spanning trees, and so $G$ does not have exactly two spanning trees.

    \item This is false. Suppose $G$ is any planar graph. As we saw in a previous part of this question, $G$ has a vertex $v$ such that $d_v \leq 5$. So if we remove the set of neighbours $N(v)$ of $v$ from $G$, this will render it disconnected. So we have removed a set of five vertices and thereby disconnected the graph, and thus it cannot be six-connected.
    
    \item Whether this is true or false depends on if you interpret ``tree'' to mean ``finite tree'' or ``any tree''. It is false for finite trees, since all finite trees have leaves.\sidenote[][]{We gave a proof of this fact in one of the earliest lectures.} However, there definitely are $k$-regular trees for every $k$ if you allow infinite trees -- to get a two-regular tree, just turn $\Z$ into a graph by adding an edge between $i$ and $i+1$ for every $i$. 
\end{enumerate}

\end{document}
